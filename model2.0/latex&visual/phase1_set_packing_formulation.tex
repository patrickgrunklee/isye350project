\documentclass[11pt]{article}
\usepackage{amsmath}
\usepackage{amssymb}
\usepackage{geometry}
\usepackage{booktabs}
\usepackage{enumitem}

\geometry{margin=1in}

\title{Phase 1: Set Packing Optimization Model\\
Mathematical Formulation}
\author{ISyE 350 - InkCredible Supplies Warehouse Optimization}
\date{}

\begin{document}

\maketitle

\section{Problem Description}

The Set Packing Optimization Model solves the optimal packing configuration for each \textbf{(facility, storage type)} combination. The goal is to determine how many packages of each SKU to place on a single shelf to maximize utilization while ensuring all SKUs are represented.

\subsection{Model Context}

\begin{itemize}[itemsep=0pt]
    \item \textbf{Scope:} One optimization model is solved independently for each (facility, storage type) pair
    \item \textbf{Facilities:} Columbus, Sacramento, Austin
    \item \textbf{Storage Types:} Bins, Racking, Pallet, Hazmat
    \item \textbf{Output:} Optimal package counts per shelf that minimize wasted space
    \item \textbf{Constraint:} Each SKU using a storage type must have at least 1 package on the shelf
\end{itemize}

\section{Mathematical Formulation}

For a given facility $f$ and storage type $st$, the complete optimization model is:

\vspace{0.5cm}

\noindent\textbf{SETS AND INDICES}

\begin{align*}
i \in S \quad &\text{SKUs assigned to storage type } st \text{ at facility } f
\end{align*}

\vspace{0.3cm}

\noindent\textbf{PARAMETERS}

\begin{align*}
v_i &\quad \text{Volume per package of SKU } i \text{ (cubic feet)} \\
w_i &\quad \text{Weight per package of SKU } i \text{ (lbs)} \\
V &\quad \text{Shelf volume capacity (cubic feet)} \\
W &\quad \text{Shelf weight capacity (lbs)} \\
N &\quad \text{Maximum number of packages per shelf}
\end{align*}

\vspace{0.3cm}

\noindent\textbf{DECISION VARIABLES}

\begin{align*}
x_i &\in \mathbb{Z}_+, \quad x_i \geq 1 \quad \forall i \in S \\
&\text{Number of packages of SKU } i \text{ to place on one shelf}
\end{align*}

\vspace{0.3cm}

\noindent\textbf{OBJECTIVE FUNCTION}

\begin{equation}
\max \quad Z = \frac{1}{2} \left( \frac{\sum_{i \in S} v_i x_i}{V} + \frac{\sum_{i \in S} w_i x_i}{W} \right)
\end{equation}

Maximize average shelf utilization (volume and weight).

\vspace{0.3cm}

\noindent\textbf{CONSTRAINTS}

\begin{align}
\sum_{i \in S} v_i x_i &\leq V && \text{(Volume capacity)} \\
\sum_{i \in S} w_i x_i &\leq W && \text{(Weight capacity)} \\
\sum_{i \in S} x_i &\leq N && \text{(Maximum package count)} \\
x_i &\geq 1 \quad \forall i \in S && \text{(SKU representation)}
\end{align}

\subsection{Constraint Descriptions}

\begin{itemize}
    \item \textbf{Volume Capacity (2):} Total volume used cannot exceed shelf volume capacity
    \item \textbf{Weight Capacity (3):} Total weight cannot exceed shelf weight capacity
    \item \textbf{Maximum Package Count (4):} Total packages cannot exceed maximum allowed per shelf
    \item \textbf{SKU Representation (5):} Each SKU must have at least one package on the shelf
\end{itemize}

\section{Model Implementation Details}

\subsection{GAMSPy Implementation}

The model is implemented in GAMSPy (Python interface to GAMS) using:

\begin{itemize}
    \item \textbf{Problem Type:} Mixed Integer Programming (MIP)
    \item \textbf{Solver:} Default GAMSPy MIP solver
    \item \textbf{Optimization Sense:} Maximize
    \item \textbf{Variables:} Integer-valued decision variables with lower bound of 1
\end{itemize}

\subsection{Model Statistics (per facility-storage type combination)}

\begin{itemize}
    \item \textbf{Number of SKUs:} Varies by storage type (typically 2-8 SKUs)
    \item \textbf{Decision Variables:} $|S|$ integer variables
    \item \textbf{Constraints:} 3 inequality constraints + lower bounds
    \item \textbf{Solution Time:} Typically $<$ 1 second per configuration
\end{itemize}

\subsection{Output Metrics}

For each solved configuration, the model reports:

\begin{itemize}
    \item Optimal number of packages per SKU
    \item Volume utilization percentage: $\frac{\sum v_i x_i^*}{V} \times 100\%$
    \item Weight utilization percentage: $\frac{\sum w_i x_i^*}{W} \times 100\%$
    \item Item count utilization: $\frac{\sum x_i^*}{N} \times 100\%$
    \item Bottleneck constraint (Volume, Weight, or Item count)
\end{itemize}

\section{Example Instance}

\subsection{Sample Data}

Consider Sacramento facility, Pallet storage type:

\textbf{SKUs using Pallet storage:} SKUD2, SKUD3, SKUC1

\textbf{Shelf Constraints:}
\begin{itemize}
    \item Volume capacity: $V = 510$ cu ft
    \item Weight capacity: $W = 600$ lbs
    \item Max items/shelf: $N = 4$ packages
\end{itemize}

\textbf{Package Data:}
\begin{center}
\begin{tabular}{lccc}
\toprule
SKU & Volume $v_i$ (cu ft) & Weight $w_i$ (lbs) & Storage Type \\
\midrule
SKUD2 & 20.0 & 75.0 & Pallet \\
SKUD3 & 18.0 & 35.0 & Pallet \\
SKUC1 & 22.2 & 60.0 & Pallet \\
\bottomrule
\end{tabular}
\end{center}

\subsection{Optimal Solution}

The optimization would determine values $x_{\text{SKUD2}}^*$, $x_{\text{SKUD3}}^*$, $x_{\text{SKUC1}}^*$ that:
\begin{itemize}
    \item Maximize average utilization
    \item Satisfy all capacity constraints
    \item Ensure each SKU has at least 1 package
\end{itemize}

\section{Connection to Phase 2}

The output of Phase 1 (packing configurations) serves as input to Phase 2 (multiperiod warehouse expansion model):

\begin{itemize}
    \item \textbf{Input to Phase 2:} File \texttt{packing\_configurations.csv} containing optimal $x_i^*$ values
    \item \textbf{Usage:} Phase 2 uses these package counts to compute required shelf quantities over time
    \item \textbf{Benefit:} Ensures realistic shelf utilization in capacity planning
\end{itemize}

\section{Key Insights}

\subsection{Bottleneck Analysis}

The model identifies which constraint limits shelf utilization:
\begin{itemize}
    \item \textbf{Volume bottleneck:} Shelf fills up by volume before reaching weight/item limits
    \item \textbf{Weight bottleneck:} Heavy items reach weight limit before volume limit
    \item \textbf{Item count bottleneck:} Package capacity constraint is most restrictive
\end{itemize}

\subsection{Practical Implications}

\begin{enumerate}
    \item \textbf{Space Efficiency:} Identifies optimal mix of SKUs to minimize wasted shelf space
    \item \textbf{SKU Diversity:} Ensures all SKUs are accessible on each shelf (operational requirement)
    \item \textbf{Facility-Specific:} Different facilities have different shelf dimensions, leading to different optimal configurations
    \item \textbf{Storage-Type-Specific:} Bins, racks, and pallets have vastly different capacities and optimal mixes
\end{enumerate}

\end{document}
